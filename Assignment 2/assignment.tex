\documentclass[titlepage]{article}

\usepackage{graphicx}
\usepackage{lastpage}
\usepackage{fancyhdr}
\usepackage{hyperref}
\usepackage{amsmath}
\usepackage{tcolorbox}
\usepackage{makecell}
\usepackage{cancel}
\usepackage{geometry}
\usepackage{listings}
\usepackage{enumitem}

\hypersetup{
    colorlinks=true,
    linkcolor=blue, 
    filecolor=magenta,
    urlcolor=blue
}

\pagestyle{fancy}
\fancyhf{}
\renewcommand{\headrulewidth}{0pt}
\rfoot{\thepage \hspace{1pt} of \pageref{LastPage}}

\begin{document}


\begin{titlepage}
    \begin{center}
        \Huge
        \vspace*{3cm}
        \textbf{  ADVD \\  \quad Assignment 2} \\
        \quad EEE F313\\[3ex]
        \huge
        \quad Problem set 14 \\[2ex]
        \quad\includegraphics[scale=0.75]{BITS_Pilani-Logo.png}
        \vfill
        \huge
        \quad Sai Kartik (2020A3PS0435P)\\\quad Rajeev Rajagopal (2020A3PS1237P) \\ \quad Manpreet Singh (2020A3PS0419P)
    \end{center}
\end{titlepage}

\newgeometry{left=2cm,bottom=2cm,right=2cm,top=2cm}

\pagenumbering{roman}

\tableofcontents
\listoffigures
\listoftables
\newpage
\pagenumbering{arabic}
\setcounter{page}{1}

\section {Problem statement}
\begin{tcolorbox}
    Design a two stage CMOS OPAMP as shown in the figure
    \begin{center}
        \includegraphics[width=0.5\textwidth]{questionpng.png}
    \end{center}
    \begin{enumerate}[label=(\alph*)]
        \item Analysis of all equations of your design, with a systematic derivation of all transistors W/L ratios and spectre simulation of circuit for the following specifications:
              \begin{enumerate}[label=(\roman*)]
                  \item Open loop gain (DC gain) $\geq$ 110dB
                  \item Phase margin $\approx 60^{\circ}$
                  \item Power dissipation $\leq$ 1mW
              \end{enumerate}
        \item Show a biasing circuitry to bias all the voltages in your design (except the input)
        \item Calculate and plot the following parameters for your OPAMP: DC gain, Bode plot for AC gain and phase, ICMR plot, slew rate, Differential Output voltage swing (dc+transient), power consumption, and input and output offset voltage.
    \end{enumerate}
\end{tcolorbox}
\newpage

\section{Design}
\subsection{Circuit diagram}
\begin{figure}[ht]
    \centering
    \includegraphics[scale = 0.505]{circ.jpg}
    \caption{Circuit diagram}
    \label{fig:circuit}
\end{figure}
\subsection{W/L calculations}
\begin{itemize}
    \item Current will be same in all the branches of the circuit. Approximate value of $|\text{V}_{\text{DS}}|$ was calculated accordingly.
    \item PMOS overdrive was set at 0.36V and NMOS overdrive at 0.3V because of the lesser mobility of PMOS, which hence needs a higher overdrive voltage.
    \item To obtain W/L of all the MOSFETs:
          \begin{itemize}
              \item  The MOSFETs were taken to a separate SPICE file.
              \item  Since the L value is constant, the value for W was found by parametrically sweeping accordingly to match requirements
          \end{itemize}
\end{itemize}
\subsubsection{W/L table}
\begin{table}[ht]
    \centering
    \begin{tabular}{|c|c|c|}
        \hline
        MOSFET                  & W            & W/L \\
        \hline
        M1 \& M2                & 5.6  $\mu$   & 16  \\
        M3 \& M4                & 5.6  $\mu$   & 99  \\
        M5                      & 65.8 $\mu$   & 188 \\
        M6 \& M7  \& M10 \& M12 & 2.8  $\mu$   & 8   \\
        M8 \&M9                 & 56  $\mu$    & 160 \\
        M11                     & 16.55  $\mu$ & 47  \\
        \hline
    \end{tabular}
    \caption{Table for W/L values}
\end{table}
\newpage
\section{Plots and results}

\subsection{Bode plot for AC gain and phase}
\begin{figure}[ht]
    \centering
    \includegraphics[scale = 0.3]{gain.png}
    \caption{AC gain at }
    \label{fig:acgain}
\end{figure}
\noindent UGB from the given circuit is 218.776 MHz. \\With the current circuitry and number of stages the maximum gain we are able to achieve is 66.48dB at 27 $^{\circ}$C. The gain can be increased by adding more stages at the expense of power consumption.
\subsection{ICMR plot}
\begin{figure}[ht]
    \centering
    \includegraphics[scale = 1.3]{icmr.png}
    \caption{ICMR plot}
    \label{fig:icmr}
\end{figure}
\newpage
\subsection{Slew rate}
\begin{figure}[ht]
    \centering
    \includegraphics[scale = 0.5]{slew.jpg}
    \caption{Slew rate}
    \label{fig:sr}
\end{figure}
\newpage
\subsection{Differential output voltage swing}
\begin{figure}[ht]
    \centering
    \includegraphics[scale = 0.5]{dov.jpg}
    \caption{Differential output voltage swing}
    \label{fig:diff}
\end{figure}
\subsection{Power consumption}
Power consumed by the circuit is given by the product: Total current drawn $\times$ $\text{V}_{\text{dd}}$ \\
From the circuit simulated by LTSpice, we have the total current drawn as $\text{I}_{\text{tot}} = 329.4 \mu \text{A}$ and $\text{V}_{dd} = 2.5V$ . \\
Hence  the total power consumption: $\text{I}_{\text{tot}}\times\text{V}_{\text{dd}} = 329.24\mu \text{A} \times 2.5 \text{V} = 0.8321 \text{mW} \leq 1 \text{mW}$
\subsection{Input and output offset voltage}
The input is an AC voltage of 30 $\mu$ sinusoid with a frequency of 1kHz. The DC offset voltage given is 1.858V.
\subsubsection{Input offset voltage}
The input offset voltage is negligible due to the circuit being systematic at the input nodes
\newpage
\subsubsection{Output offset voltage}
Output offset voltage($\text{V}_{\text{oo}}$) is defined as follows: Output voltage of the op-amp when both inputs are 0. $\text{V}_{\text{oo}}$ is due to dissimilarities in the transistors and due to mismatch in resistor values in the internal circuitry in the op-amp.
For this particular calculation, input was set as 0 and transient analysis was performed, which gives us the graph below for small time periods. For larger time periods, a steep fall was observed \\[2ex]
The output offset voltage is determined to be 38.089 mV.
\begin{figure}[ht]
    \centering
    \includegraphics[scale = 0.5]{oov.jpg}
    \caption{Output offset voltage}
    \label{fig:offset}
\end{figure}
\section{Design Challenges}
\begin{enumerate}
    \item It was very tedious to match the W/L values from both hand calculations and SPICE simulations. Trial and error was used to eventually reach an optimal value of W/L for each transistor.
    \item Obtaining parameters for a 2-stage op-amp was very tricky because of various factors like intermediate poles and separate biasing voltages of all stages
    \item No option of a pole and zero plot is available in LTSpice. Individual bode plots had to be used.
\end{enumerate}

\end{document}